\secput{solution}{Range Coalescing}

In this section we describe how an iterated predecessor query can be 
satisfied cache-obliviously using $O(\log_{B+1} n + k/B)$ 
memory transfers using a \defn{range coalescing} data structure.  
We also describe how a range coalescing data structure is built 
cache-obliviously from a set of sorted lists $L_1 \ldots L_k$ each of size $n$ 
using only $O(nk/B)$ memory transfers with high probability.  

%\subsection*{Structure}

Let $H$ be a range coalescing data structure built from a set of $k$ $n$-length
sorted lists $L_1 \ldots L_k$.  $H$ is composed of $n$ bins, each of size $O(k)$,
which partition the space of possible query values using a sorted list of $n$ 
splitters $S$, as depicted in \figref{range_coalescing}.  \figref{coalesced_bin}
illustrates how a bin concatenates $k$ sequences of elements, each of which is a 
subsequence of each constituent list from $\{L_1 \ldots L_k\}$.  
The first element of the $i$th subsequence in the $j$th bin is the predecessor
of the splitter $S_j$ in $L_i$ and is strictly smaller than it, a fact that we will 
exploit to implicitly denote the beginning of each subsequence.  

\begin{lemma}
A range coalescing data structure $H$ built from a set of $k$ $n$-length
sorted lists $L_1 \ldots L_k$ consumes $O(nk)$ space.
\end{lemma}
\begin{proof}
The elements from all $k$ $n$-length lists $L_1 \ldots L_k$ are partitioned 
evenly into $n$ bins.  In addition, each bin has exactly one element for each of 
the $k$ lists which is smaller in value than the splitter for the bin.  Thus, each
of the $n$ bins has $O(k)$ elements.
\end{proof}


\subsection*{Iterated predecessor queries}

This section describes the process by which a range coalescing data structure
answers iterated predecessor queries and demonstrates that the process consumes
$O(\log_{B+1} n + k/B)$ memory transfers with high probability.  \figref{query}
gives pseudocode for the procedure \proc{query}, which takes a range
coalescing data structure $H$ and a query $q$ and returns an ordered list of results which
correspond to the predecessors of $q$ for each constituent list in $\{ L_1 \ldots L_k \}$.

\begin{figure}[t] 
  \begin{codebox*}
    \Procname{\proc{query}$(H,q)$} 
    \li $\langle D, s \rangle \gets \proc{vEB}(\attrib{H}{S},q)$ \lilabel{find_splitter}
    \li $j \gets 1$
    \li \For $i \gets 1$ \To $\attrib{D}{\id{size}} - 1$ \Do
    \li   \If $D_i \leq q$ \Do
    \li     $\id{output}_j \gets D_i$
          \End
    \li   \If $D_{i+1} < s$ \Do
    \li     $j \gets j + 1$
          \End
        \End
    \li \Return $\id{output}$
  \end{codebox*}
\caption{Pseudocode of the \proc{query} method for a range coalescing data
structure $H$.  $H$ contains a sorted array $S$ of splitters organized using
a van Emde Boas layout~\cite{BenderDeFa00}.  The function \proc{vEB} returns 
a bin $D$, organized as an array, and a splitter $s$, which is the predecessor 
of $q$ in $S$.  The bin $D$ is walked in a linear fashion, overwriting potential
predecessors in the $\id{output}$ array and incrementing the output position whenever
the subsequence of the next list begins.  The $j$th subsequence begins with the
one and only element from $L_j$ that is smaller than the splitter $s$ and each
bin is appended with $-\infty$ to handle the boundary condition when $D_{i+1}$ is
compared with the splitter $s$. }
\label{fig:query} 
\end{figure}

While it may be that the function \proc{query} is correct by inspection, we leave
nothing to chance and prove it here.

\begin{lemma}
Given a range coalescing data structure
$H$ and a query value $q \in \mathbb{R}$, the function \proc{query($H$,$q$)} 
returns the correct answer.
\end{lemma}
\begin{proof}
Consider the $j$th bin, with corresponding splitter $S_j$, which is used to satisfy
all queries $q \in [S_j, S_{j+1})$.  By construction, the $j$th bin contains all
elements $\{q \in \cup_{i=1}^{k} L_i$ s.t. $q \in [S_j, S_{j+1}) \}$ in addition to 
the predecessor of $S_j$ from each list in $\{ L_1 \ldots L_k \}$.  
Thus, the $j$th bin contains the $k$ correct
answers --- the predecessors of $q$ in each constituent list in $\{ L_1 \ldots L_k \}$.  
We also see that each subsequence has exactly one element that is less
than the splitter $S_j$, which allows us to know which subsequence we are in during
the course of the scan.  Furthermore, since the subsequences are stored in sorted
order, we know that the predecessor results are the largest elements less than or
equal to $q$ in each subsequence.
\end{proof}

Now we bound the number of memory transfers incurred by \proc{query} by walking 
through the pseudocode in \figref{query}.

\begin{theorem}
An iterated predecessor query \proc{query}$(H,q)$ on a range coalescing data 
structure $H$ built from a set of $k$ $n$-length sorted lists $L_1 \ldots L_k$ 
incurs $O(\log_{B+1} n + k/B)$ memory transfers on a processor with a cache 
size of $M = \Omega(B)$ blocks.
\end{theorem}
\begin{proof}
We use the cache-oblivious search tree structure described by Bender, 
Demaine and Farach-Colton~\cite{BenderDeFa00} on \liref{find_splitter} of 
\figref{query} to find the bin corresponding to the predecessor in the 
sorted $n$-length splitter list $S$ using $O(\log_{B+1}n)$ memory transfers.
After we find the splitter and the corresponding bin $D$, we merely scan through
$D$ once and write out the answers in a continuous stream to the array $\id{output}$.
Thus, we incur a read stream and a write stream, each of which is $O(k)$ elements
and $O(k/B)$ memory transfers.
\end{proof}

